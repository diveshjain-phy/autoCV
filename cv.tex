%----------------------------------------------------------------------------------------
%	DOCUMENT DEFINITION
%----------------------------------------------------------------------------------------

\documentclass[a4paper,12pt]{article}

%----------------------------------------------------------------------------------------
%	PACKAGES
%----------------------------------------------------------------------------------------
\usepackage{url}
\usepackage{parskip} 	

%other packages for formatting
\RequirePackage{color}
\RequirePackage{graphicx}
\usepackage[usenames,dvipsnames]{xcolor}
\usepackage[scale=0.9]{geometry}

%tabularx environment
\usepackage{tabularx}

%for lists within experience section
\usepackage{enumitem}

% centered version of 'X' col. type
\newcolumntype{C}{>{\centering\arraybackslash}X} 

%to prevent spillover of tabular into next pages
\usepackage{supertabular}
\usepackage{tabularx}
\newlength{\fullcollw}
\setlength{\fullcollw}{0.47\textwidth}

%custom \section
\usepackage{titlesec}				
\usepackage{multicol}
\usepackage{multirow}

%CV Sections inspired by: 
%http://stefano.italians.nl/archives/26
\titleformat{\section}{\Large\scshape\raggedright}{}{0em}{}[\titlerule]
\titlespacing{\section}{0pt}{10pt}{10pt}

%for publications
\usepackage[style=authoryear,sorting=ynt, maxbibnames=2]{biblatex}

%Setup hyperref package, and colours for links
\usepackage[unicode, draft=false]{hyperref}
\definecolor{linkcolour}{rgb}{0,0.2,0.6}
\hypersetup{colorlinks,breaklinks,urlcolor=linkcolour,linkcolor=linkcolour}
\addbibresource{citations.bib}
\setlength\bibitemsep{1em}

%for social icons
\usepackage{fontawesome5}

% job listing environments
\newenvironment{jobshort}[2]
    {
    \begin{tabularx}{\linewidth}{@{}l X r@{}}
    \textbf{#1} & \hfill &  #2 \\[3.75pt]
    \end{tabularx}
    }
    {
    }

\newenvironment{joblong}[2]
    {
    \begin{tabularx}{\linewidth}{@{}l X r@{}}
    \textbf{#1} & \hfill &  #2 \\[3.75pt]
    \end{tabularx}
    \begin{minipage}[t]{\linewidth}
    \begin{itemize}[nosep,after=\strut, leftmargin=1em, itemsep=3pt,label=--]
    }
    {
    \end{itemize}
    \end{minipage}    
    }

%----------------------------------------------------------------------------------------
%	BEGIN DOCUMENT
%----------------------------------------------------------------------------------------
\begin{document}

% non-numbered pages
\pagestyle{empty} 

%----------------------------------------------------------------------------------------
%	TITLE
%----------------------------------------------------------------------------------------

\begin{tabularx}{\linewidth}{@{} C @{}}
\Huge{Divesh Jain} \\[7.5pt]
\href{https://github.com/diveshjain-phy}{\raisebox{-0.05\height}\faGithub\ diveshjain-phy} \ $|$ \ 
\href{https://linkedin.com/in/diveshjain-phy}{\raisebox{-0.05\height}\faLinkedin\ diveshjain-phy} \ $|$ \ 
\href{mailto:diveshjain.phy@gmail.com}{\raisebox{-0.05\height}\faEnvelope \ diveshjain.phy@gmail.com} \ $|$ \ 
\href{tel:+91-9337251960}{\raisebox{-0.05\height}\faMobile \ +91-9337251960} \\
\end{tabularx}

%----------------------------------------------------------------------------------------
% EXPERIENCE SECTIONS
%----------------------------------------------------------------------------------------

%Interests/ Keywords/ Summary
\section{Summary}
Dedicated and results-driven Ph.D. student with extensive experience in software development and research within the field of astrophysics. 

%Experience
\section{Work Experience}

\begin{joblong}{Ph.D. in Physics}{July 2021 - Present}
National Centre for Radio Astrophysics - TIFR, Pune, India
\item Thesis focus: Design and development of simulation of patchy reionization signatures on the Cosmic Microwave Background (CMB) using advanced statistical and computational methods 
\item Key Coursework: Astronomical Techniques I & II (with focus on Radio Astronomy) , Mathematical Methods in Physics (with focus on Statistical Techniques)
\end{joblong}


\begin{joblong}{M.Sc. in Physics}{July 2018 - June 2021}
National Centre for Radio Astrophysics - TIFR, Pune, India
\end{joblong}
  
\begin{joblong}{B.Tech in Electrical & Electronics Engineering}{July 2013-July 2017}
Veer Surendra Sai University of Technology (formerly UCE), Burla, India
\end{joblong}

%Projects
\section{Projects}

\begin{tabularx}{\linewidth}{ @{}l r@{} }
\textbf{Super resolution Reionization simulation} & \hfill Feb 2024 - Ongoing  \\[3.75pt]
\multicolumn{2}{@{}X@{}}{
* Conceptualized and led the development of a cGAN architecture for upsampling low-resolution 3D ionization fields. 
* Achieved 2x resolution while reducing computational time by 8x. Results in 54% more accuracy than traditional methods. 
* Successfully generated 100+ diverse scenarios for comprehensive analysis, ensuring accuracy and fidelity of model.
}  \\
\end{tabularx}

\begin{tabularx}{\linewidth}{ @{}l r@{} }
\textbf{Modelling and forecasting impact of Reionization on CMB} & \hfill July 2021- Jan 2024  \\[3.75pt]
\multicolumn{2}{@{}X@{}}{
* Initiated and led the project to understand the impact of reionization on CMB observations using realistic simulation techniques.
* Developed the most robust algorithm of the simulation through a Python-FORTRAN implementation.
* Conducted 50+ MCMC simulations to robustly understand the impact of reionization on CMB observations in diverse scenarios. 
* Published 3 papers as lead author in MNRAS.
}  \\
\end{tabularx}

\begin{tabularx}{\linewidth}{ @{}l r@{} }
\textbf{Imaging and Analysis of Radio Phoenix sources} & \hfill July 2020 - December 2020  \\[3.75pt]
\multicolumn{2}{@{}X@{}}{
* Contributed to a python-based CASA pipeline with sub-banding to reduce and image GMRT GWB Data at Band 3 and Band 5 for Galaxy Cluster A3017. Discovered a candidate for Radio Phoenix.
* Developed a Python-based implementation of an adiabatic compression model to ascertain the stage of life of Radio Phoenix in Abell 1914. Used the chi-square fitting techniques to affirm the current stage of the radio source. 
}  \\
\end{tabularx}

\begin{tabularx}{\linewidth}{ @{}l r@{} }
\textbf{Transient stability enhancement of a Multi-machine system using Renewable Energy Resources} & \hfill May 2016 - March 2017  \\[3.75pt]
\multicolumn{2}{@{}X@{}}{
* Developed an extensive MATLAB framework to simulate, study, and enhance transient stability of multi-machine systems.
* Contributed in designing a control structure that improves the system response by 35% through coordinated control than the traditional techinques. 
* Published a paper showcasing the project’s findings and contributions.
}  \\
\end{tabularx}


%----------------------------------------------------------------------------------------
%	EDUCATION
%----------------------------------------------------------------------------------------
\section{Education}
\begin{tabularx}{\linewidth}{@{}l X@{}}	
2021 - present & PhD (Physics) at \textbf{National Centre for Radio Astrophysics - TIFR, Pune, India} \\

2018 - 2021 & Master's Degree (Physics) at \textbf{National Centre for Radio Astrophysics - TIFR, Pune, India}  \\ 

2013 - 2017 & Bachelor's Degree (Electrical & Electronics Engineering) at \textbf{Veer Surendra Sai University of Technology, Burla, India} \\ 
\end{tabularx}

%----------------------------------------------------------------------------------------
%	PUBLICATIONS
%----------------------------------------------------------------------------------------
\section{Publications}
\printbibliography[heading=none]

%----------------------------------------------------------------------------------------
%	SKILLS
%----------------------------------------------------------------------------------------
\section{Skills}
\begin{tabularx}{\linewidth}{@{}l X@{}}
Languages &  \normalsize{Python, C/C++, FORTRAN, MATLAB}\\
Software  &  \normalsize{Linux, CASA, Pytorch, Cobaya, Git, Bash, LaTeX, Vim}\\  
Certifications & \normalsize{Post Graduate Degree in Cyber Security, Great Learning (Jan 2023 - July 2023)} \\
\end{tabularx}

\vfill
\center{\footnotesize Last updated: \today}

\end{document}
